% Nikolai Nielsens "Fysiske Fag" preamble
\documentclass[a4paper,10pt]{article} 	% A4 papir, 10pt størrelse
\usepackage[danish]{babel}
\usepackage{Nikolai} 					% Min hjemmelavede pakke
\usepackage[dvipsnames]{xcolor}

% Margen
\usepackage[margin=1in]{geometry}

% Max antal kolonner i en matrix. Default er 10
%\setcounter{MaxMatrixCols}{20}

% Hvor dybt skal kapitler labeles?
%\setcounter{secnumdepth}{4}	
%\setcounter{tocdepth}{4}

% Hvilket nummer skal der startes med i sections? (n-1)
%\setcounter{section}{0}	

% Til nummerering af ligninger. Så der står (afsnit.ligning) og ikke bare (ligning)
\numberwithin{equation}{section}


% Header
%\usepackage{fancyhdr}
%\head{}
%\pagestyle{fancy}

%Titel
\title{Lattice classification}
\author{}
\date{}

\begin{document}
	\selectlanguage{danish}
	\maketitle

	\section{Face Centered cubic. Primitive}
	$a_1 = (1/2, 1/2, 0),\ a_2 = (1/2, 0, 1/2),\ a_3 = (0, 1/2, 1/2)$
	
	Internal angles all 60 degrees ($\cos \theta = 1/2$). 45 degrees with respect to two cardinal axis ($\cos \theta = \sqrt{2}/2$), Orthogonal with respect to last. All lengths equal ($\sqrt{2}/2$)
	
	\section{Body centered cubic. Primitive}
	$ a_1 = (1,0,0),\ a_2 = (0,1,0),\ a_3 = (1/2,1/2,1/2) $
	
	Internal angles: $a_1$ and $a_2$ are orthogonal. $a_1$ or $a_2$ with $a_3$ has $\cos \theta = \sqrt{3}/3 $ (roughly 54.74 degrees). Two vectors have length 1, last has length $\sqrt{3}/2$.
	
	\section{Tetragonal Body centered}
	Increase z-coordinate on $a_3$. Say $a_3 = (a/2,a/2,b/2) $. Then $\cos \theta = a/\sqrt{2a^2+b^2} $, and $|a_3| = \frac{\sqrt{2a^2+b^2}}{2}$, or $ \cos \theta = |a_1|/2|a_3|$:
	\begin{align}
		a_1 &= (a,0,0), \quad a_2 = (0,a,0), \quad a_3 = (a/2,a/2,b/2) \\
		|a_1| &= a = |a_2|, \quad |a_3| = \frac{\sqrt{2a^2+b^2}}{2}, \\
		\cos \theta_{12} &= 0, \quad \cos \theta_{23} = \cos \theta_{31} = \frac{|a_1|}{2|a_3|}  = \frac{|a_2|}{2 |a_3|}
	\end{align}
	
	\section{Tetragonal Face centered}
	\begin{align}
		a_1 &= (a/2,a/2,0), \quad a_2 = (a/2,0,b/2), \quad a_3 = (0,a/2,b/2),\\
		|a_1| &= \frac{\sqrt{2}}{2} a, \quad |a_2| = \frac{\sqrt{a^2+b^2}}{2} = |a_3|, \\
		\cos \theta_{12} & = \frac{|a_1|}{2|a_2|} = \cos \theta_{31} = \frac{|a_1|}{2 |a_3|}, \quad \cos \theta_{23} = \frac{b^2}{a^2+b^2} = \frac{2a_2^2 - a_1^2}{2a_2^2} = \frac{2a_3^2 - a_1^2}{2a_3^2}
	\end{align}
	\section{Tetragonal base centered}
	\begin{align}
		a_1 &= (a/2,a/2,0), \quad a_2 = (0,a,0), \quad a_3 = (0,0,b) \\
		|a_1| &= \frac{\sqrt{2}}{2} a, \quad |a_2| = a, \quad |a_3| = b \\
		\cos \theta_{12} &= \frac{\sqrt{2}}{2}, \quad \cos \theta_{31} = \cos\theta_{23} = 0
	\end{align}
	
	\section{Orthorhombic}
	Let's assume the conventional unit cell has $a_1 = (a,0,0),\ a_2 = (0,b,0),\ a_3 = (0,0,c)$
	
	\subsection{Body centered}
	$a_3 = (a/2, b/2, c/2)$. Lengths: $|a_1| = a,\ |a_2| = b, |a_3| = \frac{\sqrt{a^2+b^2+c^2}}{2}$. Angles: 
	
	\begin{equation}
		\cos \theta_{12} = 0,\ \cos \theta_{31} = \frac{a}{\sqrt{a^2+b^2+c^2}} = \frac{|a_1|}{2|a_3|},\ \cos \theta_{23} = \frac{b}{\sqrt{a^2+b^2+c^2}} = \frac{|a_2|}{2|a_3|}
	\end{equation}
	And the spacing is $ c^2 = 4a_3^2-a_1^2-a_2^2 $
	
	These two last are also angles with respect to x-axis and y-axis respectively. Angle with z is $\cos\theta_{3z} = \frac{c}{\sqrt{a^2+b^2+c^2}} = 2(4a_3^2-a_1^2-a_2^2)/|a_3|$
	
	\subsection{Face centered}
	$ a_1 = (a/2,b/2,0), \ a_2 = (a/2,0,c/2), \ a_3 = (0,b/2,c/2) $.
	
	\begin{align} 
	|a_1| &= \frac{\sqrt{a^2+b^2}}{2}, \ |a_2| = \frac{\sqrt{a^2+c^2}}{2}, \ |a_3| = \frac{\sqrt{b^2+c^2}}{2} \\
	\cos \theta_{12} &= \frac{a^2}{\sqrt{a^2+c^2} \cdot \sqrt{a^2+b^2}},\ \cos \theta_{31} = \frac{b^2}{\sqrt{b^2+c^2} \cdot \sqrt{a^2+b^2}},\ \cos \theta_{23} = \frac{c^2}{\sqrt{a^2+c^2} \cdot \sqrt{b^2+c^2}} 
	\end{align}
	
	And the spacings are
	\begin{equation}\label{key}
		a^2 = 2 (a_1^2+a_2^2-a_3^2), \ b^2 = 2(a_1^2-a_2^2+a_3^2), \ c^2 = 2(-a_1^2+a_2^2+a_3^2)
	\end{equation}
	As such the angles can be written
	\begin{equation}\label{key}
		\cos \theta_{12} = \frac{a_1^2+a_2^2-a_3^2}{2 \cdot |a_1| \cdot |a_2|}, \quad
		\cos \theta_{31} = \frac{a_1^2-a_2^2+a_3^2}{2 \cdot |a_1| \cdot |a_3|}, \quad
		\cos \theta_{23} = \frac{-a_1^2+a_2^2+a_3^2}{2 \cdot |a_2| \cdot |a_3|}
	\end{equation}
	
	
	\subsection{Base centered}
	$ a_1 = (a,0,0),\ a_2 = (a/2,b/2,0),\ a_3 = (0,0,c) $.
	
	Spacings and angles
	\begin{equation}\label{key}
	|a_2|^2 = \frac{a^2+b^2}{4}, \ b^2 = 4a_2^2-a_1^2. \quad \cos \theta_{12} = \frac{a}{\sqrt{a^2+b^2}} = \frac{|a_1|}{2|a_2|}, \ \cos\theta_{31} = \cos \theta_{23} = 0
	\end{equation}
	
	\section{Simple monoclinic}
	\url{https://en.wikipedia.org/wiki/Monoclinic_crystal_system#/media/File:Monoclinic.svg}

	\begin{equation}\label{key}
		|a_1| = a,\ |a_2| = b,\ |a_3| = c, \quad \cos \theta_{12} = \cos \theta_{23} = 0, \ \cos \theta_{31} \neq 0
	\end{equation}
	
	$a_1$ and $a_2$ are along x and y. $a_3 = c \cdot (\cos \theta_{31},0,\sin \theta_{31})$
	
	\section{Base centered monoclinic}
	$ a_1 = (a,0,0),\ a_2 =(a/2, b/2,0),\ a_3 = c \cdot (\cos \theta_{31},0,\sin \theta_{31}) $.
	
	\begin{equation}\label{key}
	|a_1| = a,\ |a_2| = \frac{\sqrt{a^2+b^2}}{2}, \ |a_3| = c \quad \cos \theta_{12} = \frac{|a_1|}{2|a_2|}, \ \cos \theta_{23} = \frac{a\cos\theta_{31}}{\sqrt{a^2+b^2}} = \frac{a_1 \cdot a_3}{2 \cdot |a_2| \cdot |a_3|}
	\end{equation}
	
	
	\section{Hexagonal}
	$ a_1 = (1,0,0),\ a_2 = (1/2,\sqrt{3}/2,0),\ a_3 = (0,0,a) $. $a_1$ and $a_2$ are orthogonal to $a_3$, and 60 degrees between them.
	$|a_1| = |a_2| \neq |a_3| \quad \cos \theta_{12} = 1/2, \ \cos \theta_{31} = \cos \theta_{23} = 0$
	
	\section{Triclinic}
	$|a_1| \neq |a_2| \neq |a_3|$. And $\theta_{12} \neq \theta_{31} \neq \theta_{23}$
	
	\section{Rhombohedral}
	$|a_1| = |a_2| = |a_3|$ and $\theta_{12} = \theta_{31} = \theta_{23}$ but they're not right angles

\end{document}



