% Nikolai Nielsens "Fysiske Fag" preamble
\documentclass[a4paper,11pt]{article}
\usepackage[english]{babel}
\usepackage{Nikolai}
\usepackage[dvipsnames]{xcolor}
\usepackage[margin=0.75in]{geometry}
\usepackage{wrapfig}

\usepackage{pdfpages}

\usepackage{listings}
\usepackage{color}
\definecolor{Maroon}{RGB}{175,50,53}
\definecolor{OliveGreen}{RGB}{60,128,49}
\definecolor{Orange}{RGB}{245,129,55}

\lstset{ %
	language=python,
	numbers=left,
	stepnumber=1,
	breaklines=true,
	keepspaces=true,
	showstringspaces=false, 
	tabsize=4,
	basicstyle=\footnotesize\ttfamily,
	keywordstyle=\bfseries\color{Maroon},
	commentstyle=\itshape\color{OliveGreen},
	identifierstyle=\color{blue},
	stringstyle=\color{Orange},
}

% Til nummerering af ligninger. Så der står (afsnit.ligning) og ikke bare (ligning)
\numberwithin{equation}{section}

\newcommand{\coef}[1]{_{[#1]}}

\title{Documentation for the \texorpdfstring{\texttt{cmp.py}}{cmp.py} script}
\date{}

\begin{document}
	\maketitle
	
	\section{Introduction}
	The program in this project is written in the open source programming language \href{www.python.org}{Python}, using the standard library, along with the \href{www.numpy.org}{NumPy} and \href{www.matplotlib.org}{Matplotlib} packages, with \href{www.pypi.org/project/PyQt5/}{PyQt5} being used for the GUI. The versions used in development was
	\begin{itemize}
		\item Python v3.6.5
		\item NumPy v1.14.3
		\item Matplotlib v2.2.2
		\item PyQt 5.9.2
	\end{itemize}
	The program should work with newer versions, though maybe not if they are major overhauls (say going from Numpy 1.x.y to 2.x.y). You can find instructions on how to install the different packages on their respective websites.
	
	It is however recommended that you install \href{https://www.anaconda.com/}{Anaconda}, which is a Python distribution that also includes both NumPy, Matplotlib and PyQt5, along with a wealth of other packages useful for scientific computing. The \href{http://jupyter.org/}{Jupyter} package, especially, is useful, as it allows the user to write a bunch of small scripts (called cells) in a single file (called a notebook), and run the cells individually and quickly, while sharing the workspace between cells (i.e., variables persist between cells). A quick intro to Jupyter notebooks can be found at \href{http://nbviewer.jupyter.org/github/jupyter/notebook/blob/master/docs/source/examples/Notebook/Notebook Basics.ipynb}{this link}.
	
	\section{Installation}
	As recommended we will install the Anaconda distribution. Download the software from here: \url{https://www.anaconda.com/download/}. Install the software, and when asked if you want to include Anaconda to the path (or something along those lines) say: ``Yes''. This will allow you to start Python from your command line/terminal with the command \texttt{python}.
	
	Next you will need the files 
	
	

\end{document}