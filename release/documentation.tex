% Nikolai Nielsens "Fysiske Fag" preamble
\documentclass[a4paper,11pt]{article}
\usepackage[english]{babel}
\usepackage{Nikolai}
\usepackage[dvipsnames]{xcolor}
\usepackage[margin=0.75in]{geometry}
\usepackage{wrapfig}

\usepackage{pdfpages}

\usepackage{listings}
\usepackage{color}
\definecolor{Maroon}{RGB}{175,50,53}
\definecolor{OliveGreen}{RGB}{60,128,49}
\definecolor{Orange}{RGB}{245,129,55}

\lstset{ %
	language=python,
	numbers=left,
	stepnumber=1,
	breaklines=true,
	keepspaces=true,
	showstringspaces=false, 
	tabsize=4,
	basicstyle=\footnotesize\ttfamily,
	keywordstyle=\bfseries\color{Maroon},
	commentstyle=\itshape\color{OliveGreen},
	identifierstyle=\color{blue},
	stringstyle=\color{Orange},
}

% Til nummerering af ligninger. Så der står (afsnit.ligning) og ikke bare (ligning)
\numberwithin{equation}{section}

\newcommand{\coef}[1]{_{[#1]}}

\title{Documentation for the \texorpdfstring{\texttt{cmp.py}}{cmp.py} script}
\date{}

\begin{document}
	\maketitle
	
	\section{Introduction}
	The program in this project is written in the open source programming language \href{www.python.org}{Python}, using the standard library, along with the \href{www.numpy.org}{NumPy} and \href{www.matplotlib.org}{Matplotlib} packages, with \href{www.pypi.org/project/PyQt5/}{PyQt5} being used for the GUI. The versions used in development was
	\begin{itemize}
		\item Python 3.6.5
		\item NumPy 1.14.3
		\item Matplotlib 2.2.2
		\item PyQt 5.9.2
	\end{itemize}
	The program should work with newer versions, though maybe not if they are major updates (where a package changes its leftmost version number). You can find instructions on how to install the different packages on their respective websites.
	
	It is however recommended that you install \href{https://www.anaconda.com/}{Anaconda}, which is a Python distribution that also includes both NumPy, Matplotlib and PyQt5, along with a wealth of other packages useful for scientific computing. The \href{http://jupyter.org/}{Jupyter} package, especially, is useful, as it allows the user to write a bunch of small scripts (called cells) in a single file (called a notebook), and run the cells individually and quickly, while sharing the workspace between cells (i.e., variables persist between cells). A quick intro to Jupyter notebooks can be found at \href{http://nbviewer.jupyter.org/github/jupyter/notebook/blob/master/docs/source/examples/Notebook/Notebook Basics.ipynb}{this link}.
	
	\section{Installation}
	As recommended we will install the \href{https://www.anaconda.com/download/}{Anaconda distribution}. Install the software, and when asked if you want to include Anaconda to the path (or something along those lines) say: ``Yes''. This will allow you to start Python from your command line/terminal with the command \texttt{python}.
	
	Next you will need the files \texttt{cmp.py}, \texttt{lattices.py}, \texttt{gui.py} and \texttt{notebook.ipynb}, which can be found on the course page, or at \href{https://github.com/NikolaiNielsen/Bachelor/tree/gui/release}{the following GitHub page}.
	
	\section{Usage}
	There are two main ways of starting the program. Either directly from the command line or from a Jupyter Notebook. Using the command line:
	\begin{itemize}
		\item Start the terminal/command line
		\item navigate to the directory where you downloaded the files to.
		\item type \texttt{python cmp.py} and press return.
	\end{itemize}
	or using Jupyter Notebooks:
	\begin{itemize}
		\item Start Jupyter Notebooks. Either by opening the terminal/command line and typing \texttt{jupyter notebook}, or by opening the program \texttt{anaconda-navigator}, installed with Anaconda, and then launch Jupyter Notebooks from there
		\item A browser window should have appeared. Navigate to the directory where you downloaded the files to and open the file \texttt{notebook.ipynb}.
		\item in the open cell type \texttt{run cmp.py} and press ctrl+return to run the cell.
	\end{itemize}
	
	

\end{document}