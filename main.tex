% Nikolai Nielsens "Fysiske Fag" preamble
\documentclass[a4paper,10pt]{article}
\usepackage[english]{babel}
\usepackage{Nikolai}
\usepackage[dvipsnames]{xcolor}
\usepackage[margin=1in]{geometry}

%\setcounter{MaxMatrixCols}{20}
%\setcounter{secnumdepth}{4}	
%\setcounter{tocdepth}{4}
%\setcounter{section}{0}

% Til nummerering af ligninger. Så der står (afsnit.ligning) og ikke bare (ligning)
\numberwithin{equation}{section}


% Header
%\usepackage{fancyhdr}
%\head{}
%\pagestyle{fancy}

%Titel
\title{Visualisation of concepts in condensed matter physics}
\author{Nikolai Plambech Nielsen, LPK331}
\date{2018-06-13}

\begin{document}
	
	\maketitle
	
	\tableofcontents
	
	\section{General implementation}
	All of the programs in this project will be implemented in the open source programming language \href{www.python.org}{Python}, using the standard library, along with the \href{www.numpy.org}{NumPy} and \href{www.matplotlib.org}{Matplotlib} packages. Numpy adds the necessary tools for handling arrays and linear algebra operations, whilst Matplotlib is used for plotting the results.
	
	One thing to note is that doing numerical calculations will lead to some numerical error. As such testing for numerical equality will often time yield the "wrong" result. Because of this, we employ the \texttt{isclose} function, which takes 4 parameters: \texttt{a}, \texttt{b}, \texttt{atol} and \texttt{rtol}. \texttt{a} and \texttt{b} are the numerical values to be tested for equality, \texttt{atol} is the absolute tolerance and \texttt{rtol} is the relative tolerance. The function then tests if $|\texttt{a}-\texttt{b}| \leq (\texttt{atol}+ \texttt{rtol} \D |\texttt{b}|)$, with default values for \texttt{atol} and \texttt{rtol} being $ 1 \D 10^{-8} $ and $ 1 \D 10^{-5} $ respectively.
	
	\subfile{lattices.tex}
	
	\newpage
	\subfile{reciprocal.tex}
	
\end{document}