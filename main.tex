% Nikolai Nielsens "Fysiske Fag" preamble
\documentclass[a4paper,11pt]{article}
\usepackage[english]{babel}
\usepackage{Nikolai}
\usepackage[dvipsnames]{xcolor}
\usepackage[margin=1in]{geometry}

%\setcounter{MaxMatrixCols}{20}
%\setcounter{secnumdepth}{4}	
%\setcounter{tocdepth}{4}
%\setcounter{section}{0}

% Til nummerering af ligninger. Så der står (afsnit.ligning) og ikke bare (ligning)
\numberwithin{equation}{section}

\newcommand{\coef}[1]{_{[#1]}}

% Header
%\usepackage{fancyhdr}
%\head{}
%\pagestyle{fancy}

%Titel
\title{Visualisation of concepts in condensed matter physics}
\author{Nikolai Plambech Nielsen, LPK331}
\date{2018-06-13}

\begin{document}
	
	\maketitle
	
	\tableofcontents
	
	\newpage
	\subfile{introduction.tex}
	
	\subfile{lattices.tex}
	
	\subfile{reciprocal.tex}
	
	\subfile{2D_band_structure.tex}
	
	\section{References}
	
	
	\newpage
	\appendix
	\subfile{latticeClassification.tex}
	
\end{document}