\documentclass[main.tex]{subfiles}

\begin{document}
	\section{Discussion}
	In this thesis we set out to create a suite of programs to visualise concept in the field of condensed matter physics. These concepts included crystal structures, families of lattice planes, a simulation of neutron scattering, along with band structures of two dimensional materials.
	
	The end result is a code package written in Python which accomplishes just that. It includes 4 main functions: \texttt{Lattice}, \texttt{Reciprocal}, \texttt{Scattering} and \texttt{Band\_structure}. These 4 functions take in a wealth of arguments, supplied by the user, and produce some sort of figure. Be that the crystal structure of hexagonal lattice with a one atom basis (figure \ref{fig:lattice_demo_2}), the (001) family of lattice planes for a bcc lattice (figure \ref{fig:lattice_planes}), neutron scattering on an fcc lattice (figures \ref{fig:scattering_no_systemic} and \ref{fig:scattering_systemic}) or the band structure of a monovalent two dimensional material with a high strength potential (figure \ref{fig:band_structure_strong}).
	
	
	
\end{document}